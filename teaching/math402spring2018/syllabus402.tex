% This syllabus template was created by:
% Brian R. Hall
% Associate Professor, Champlain College
% www.brianrhall.net

% Document settings
\documentclass[11pt]{article}
\usepackage[margin=1in]{geometry}
\usepackage[pdftex]{graphicx}
\usepackage{multirow}
\usepackage{setspace}
\pagestyle{plain}
\setlength\parindent{0pt}
\usepackage{hyperref}

\begin{document}

% Course information
\begin{tabular}{ l l }
  \multirow{3}{*}{} & \LARGE Math 402, Section F13 \& F14 \\\\
  & \LARGE Non-Euclidean Geometry \\\\
  & \LARGE MWF, 2-2:50p, 347 Altgeld Hall\\\\
\end{tabular}
\vspace{10mm}

% Professor information
\begin{tabular}{ l l }
  \multirow{6}{*}{} & \large Professor: Dominic Culver \\\\
  & \large email: dculver@illinois.edu \\
  & \large website: \url{https://math.illinois.edu/~dculver} under teaching \\
  & \large Office: 327 Illini Hall \\
  & \large Office Hours: TW 3-4p or by appointment \\
 % & \large (123) 867-5309 \\
\end{tabular}
\vspace{5mm}
\begin{center} The contents of this syllabus are subject to change at anytime during the semester. \\
\end{center}

% Course details
\textbf {\large \\ Official Course Description:} Historical development of geometry; includes tacit assumptions made by Euclid; the discovery of non-Euclidean geometries; geometry as a mathematical structure; and an axiomatic development of plane geometry.\\
\textbf {Prerequisite(s):} Math 241, 347, or 348; or consent of the instructor.

%\textbf {Note(s):} A minimum grade of C is required in this course to progress to COURSE. 

\textbf {Credit Hours:} 3 or 4 \\

\textbf {\large Text(s):} \textit{Geometry (with Geometry Explorer)}, Michael Hvidsten\\
This book is out of print, but the author has generously made an electronic copy available for personal use. It can be found at \url{http://new.math.uiuc.edu/public402/Hvidsten.pdf}. You could also try to purchase a used copy.\\

\textbf{\large Software:} Geometry Explorer, available at \url{ http://homepages.gac.edu/~hvidsten/gex/download-3.0.html}\\

%\textbf {Author(s):} D'Angelo and West \\

%\textbf {\large Course Objectives:} \\
%At the completion of this course, students will be able to:
%\begin{enumerate} \itemsep-0.4em
%  \item O
%  \item B
%  \item J
%  \item E
%  \item C
%  \item T
%  \item I
%  \item V
%  \item E
%  \item S
%\end{enumerate}

% I recommend using \newpage here if necessary
\textbf {\large Grade Distribution:} \\
\hspace*{40mm}
\begin{tabular}{ l l }
%Labs & 20\% \\
HW & 15\% \\
Project reports & 10\% \\
%Quizzes  & 10\% \\
Midterm Exams  & 3$\times$15\% \\
Final Exam  & 30\%
\end{tabular} \\\\

\textbf{Exam Dates:}\\
\begin{tabular}{l l}
	Exam 1 & ??\\
	Exam 2 & ??\\
	Exam 3 & ??\\
	Final Exam & TBA
\end{tabular} \\\\

%\textbf {\large Letter Grade Distribution:} \\\\
%\hspace*{40mm}
%\begin{tabular}{ l l | l l }
%\textgreater= 93.00 & A & 73.00 - 76.99 & C \\
%90.00 - 92.99 & A-  & 70.00 - 72.99 & C- \\
%87.00 - 89.99 & B+  & 67.00 - 69.99 & D+ \\
%83.00 - 86.99 & B  & 63.00 - 66.99 & D \\
%80.00 - 82.99 & B-  & 60.00 - 62.99 & D- \\
%77.00 - 79.99 & C+  & \textless= 59.99 & F \\
%\end{tabular} \\

% Course Policies. These are just examples, modify to your liking.

\pagebreak

\textbf{\large Course structure:}

\begin{itemize}
	\item \textbf{Reading assignments:}
			A reading assignment for each day of class will be posted on the course webpage ahead of time. These should be completed \emph{before class} and should be reviewed several times after. 
	\item  \textbf{Worksheets:} One class per week, typically Fridays, will be devoted to group work. You will be given a worksheet and asked to work on it in groups of 3-4. To get the most out of this activity, it is extremely important to come to class prepared, i.e. having done the reading assignments throughout the week. 
	\item \textbf{Assignments:} Homework will typically be assigned every week on Friday and due the following Friday.
	\item \textbf{Projects:} Almost every week, a \textbf{project} will be assigned, and a \textbf{report} will be due the following Monday in class. You will be asked to perform some experimentation with a mathematical phenomenon using the Geometry Explorer software. Moreover, you will supplement that with formal reasoning in order to understand the patterns or mathemat- ical laws behind said phenomenon. The assignment will be to write a report on what you have learned and how. In general, the reports should contain about a page of essay-style discussion, in addition to any formal mathematical exercises.

\end{itemize}

\textbf {\large Course Policies:}
\begin{itemize}
	\item \textbf {Grades}
		\begin{itemize}
			\item We will be using Moodle for this course. Please let me know of discrepancies in your grade as they appear throughout the term. 
		\end{itemize}
	\item \textbf {Assignments}
		\begin{itemize} 
			\item No late homework will be accepted. However, I will drop the lowest Homework grade. 
			\item You are permitted, and in fact encouraged, to discuss homework problems with your fellow classmates. However, the solutions you submit \textbf{must} be written in your own words. To elaborate, this does not simply mean the physical act of writing. One should independently write one's assignment without the assistance of outside sources (including people). You \emph{must} list your collaborators on each homework assignment. 
		\end{itemize}
	\item \textbf{Exams}:
			\begin{itemize}
				\item All exams will be closed book and closed notes. The use of calculators or other electronic devices is not permitted.
				\item There will be no make up exams. If you have a conflict, you must let me know ahead of time so that we can come to a satisfactory arrangement.
				\item There will be three midterm exams, each will be 50 minutes and given in class. 
				\item The final exam must be given at the officially assigned Final Exam slot provided by the University. In particular, a student is not permitted to take an earlier exam to accommodate travel plans. The final exam schedule will be made public on October 4, so make travel plans after this date. 
			\end{itemize}
	\item \textbf{Attendance and Absences}
		\begin{itemize}
%			\item Attendance is expected and will be taken each class. You are allowed to miss \textbf{1} class during the semester without penalty. Any further absences will result in point and/or grade deductions.
			\item Students are responsible for all missed work, regardless of the reason for absence. It is also the absentee's responsibility to get all missing notes or materials. 
		\end{itemize}
	\item \textbf{Students with disabilities}:
		\begin{itemize}
		\item Students with disabilities who require reasonable accommodations to should see me as soon as possible. In particular, any accommodation on exams must be requested at least a week in advance and will require a letter from DRES.
		\end{itemize}
	\item \textbf{Academic integrity}:
		\begin{itemize}
		\item Cheating is taken very seriously, so please don't do it. Penalties for cheating on exams, in particular, are very high, typically resulting in a 0 on the exam or an F in the class.
		\end{itemize}
\end{itemize}

\textbf{\large Tips for Success (in math classes in general):} Reading and learning mathematics takes a great deal of time and effort, and there is no shortcut to understanding the material. The following are some good habits to develop for learning mathematics. 
\begin{itemize}
	\item When doing your reading assignments, or reviewing the material, it is a good idea to try and prove the various statements on your own, referring to the textbook or your notes only when you truly get stuck. This can be done before and after your reading assignments.
	\item Make sure you understand the definitions. This implies being able to give an example and indicating the conceptual idea behind the definition. Having pictures in mind can also be helpful. When learning definitions at first, I highly recommend reading the definition, and then writing it down by hand in your own words. 
	\item Discuss the material with me, or even better, with your fellow classmates. Very often, two people have different ways of understanding the material. Discussing with your classmates will not only clarify and improve your understanding, but enrich it as well. 
	\item Take advantage of office hours!
	\item Ask yourself questions, and try to answer them. 
\end{itemize}
%
%% College Policies
%\textbf {\large Academic Honesty Policy Summary:} 
%% This should be specific to your instituition, an example is provided.
%
%\textbf{Introduction}
%
%\hspace{3mm}
%\hangindent=5mm In addition to skills and knowledge, COLLEGE/UNIVERSITY aims to teach students appropriate Ethical and Professional Standards of Conduct. The Academic Honesty Policy exists to inform students and Faculty of their obligations in upholding the highest standards of professional and ethical integrity. All student work is subject to the Academic Honesty Policy. Professional and Academic practice provides guidance about how to properly cite, reference, and attribute the intellectual property of others. Any attempt to deceive a faculty member or to help another student to do so will be considered a violation of this standard.
%
%\textbf{Instructor's Intended Purpose}
%
%\hspace{3mm}
%\hangindent=5mm The student's work must match the instructor's intended purpose for an assignment. While the instructor will establish the intent of an assignment, each student must clarify outstanding questions of that intent for a given assignment. 
%
%\textbf{Unauthorized/Excessive Assistance}
%
%\hspace{3mm}
%\hangindent=5mm The student may not give or get any unauthorized or excessive assistance in the preparation of any work.
%
%\textbf{Authorship}
%
%\hspace{3mm}
%\hangindent=5mm The student must clearly establish authorship of a work. Referenced work must be clearly documented, cited, and attributed, regardless of media or distribution. Even in the case of work licensed as public domain or Copyleft, (See: http://creativecommons.org/) the student must provide attribution of that work in order to uphold the standards of intent and authorship.
%
%\textbf{Declaration}
%
%\hspace{3mm}
%\hangindent=5mm Online submission of, or placing one's name on an exam, assignment, or any course document is a statement of academic honor that the student has not received or given inappropriate assistance in completing it and that the student has complied with the Academic Honesty Policy in that work.
%
%\textbf{Consequences}
%
%\hspace{3mm}
%\hangindent=5mm An instructor may impose a sanction on the student that varies depending upon the instructor's evaluation of the nature and gravity of the offense.  Possible sanctions include but are not limited to, the following: (1) Require the student to redo the assignment; (2) Require the student to complete another assignment; (3) Assign a grade of zero to the assignment; (4) Assign a final grade of ``F'' for the course. A student may appeal these decisions according to the Academic Grievance Procedure. (See the relevant section in the Student Handbook.) Multiple violations of this policy will result in a referral to the Conduct Review Board for possible additional sanctions. \\
%
%The full text of the Academic Honesty Policy is in the \emph{Student Handbook}.

%\newpage
% A new page is forced here. This can be moved around or altered based on how much space is needed for the course policies. Or, the \newpage can be removed. It's up to you.

%\textbf {\large Category X:}
%
%\hspace{3mm}
%\hangindent=5mm Put any other categories of information related to your college/university here. \\

% The data research disclosure can be removed if you are not into research or if you don't plan on ever using course data in publications.
%\textbf {\large Data for Research Disclosure}:
%
%Any and all results of in-class and out-of-class assignments and examinations are data sources for research and may be used in published research. All such use will always be anonymous.

%\newpage

%% Course Outline
%\textbf {\large Tentative Course Outline}:
%
%The weekly coverage might change as it depends on the progress of the class.  However, you must keep up with the reading assignments.
%
%\begin{table}[h!]
%\normalsize % The size of the table text can be changed depending on content. Remove if desired.
%\begin{tabular}{ | c | c | }
%\hline
%\textbf{Week} & \textbf{Content} \\
%\hline
%Week 1 & \begin{minipage}{.85\textwidth}
%\begin{itemize} \itemsep-0.4em
%	\vspace{1mm}
%	\item Something interesting
%	\item Reading assignment: Something interesting
%	\vspace{1mm}
%\end{itemize}
%\end{minipage} \\
%\hline
%Week 2 & \begin{minipage}{.85\textwidth}
%\begin{itemize} \itemsep-0.4em
%	\vspace{1mm}
%	\item Something interesting
%	\item Reading assignment: Something interesting
%	\vspace{1mm}
%\end{itemize}
%\end{minipage} \\
%\hline
%Week 3 & \begin{minipage}{.85\textwidth}
%\begin{itemize} \itemsep-0.4em
%	\vspace{1mm}
%	\item Something interesting
%	\item Reading assignment: Something interesting
%	\vspace{1mm}
%\end{itemize}
%\end{minipage} \\
%\hline
%Week 4 & \begin{minipage}{.85\textwidth}
%\begin{itemize} \itemsep-0.4em
%	\vspace{1mm}
%	\item Something interesting
%	\item Reading assignment: Something interesting
%	\vspace{1mm}
%\end{itemize}
%\end{minipage} \\
%\hline
%Week 5 & \begin{minipage}{.85\textwidth}
%\begin{itemize} \itemsep-0.4em
%	\vspace{1mm}
%	\item Something interesting
%	\item Reading assignment: Something interesting
%	\vspace{1mm}
%\end{itemize}
%\end{minipage} \\
%\hline
%Week 6 & \begin{minipage}{.85\textwidth}
%\begin{itemize} \itemsep-0.4em
%	\vspace{1mm}
%	\item Something interesting
%	\item Reading assignment: Something interesting
%	\vspace{1mm}
%\end{itemize}
%\end{minipage} \\
%\hline
%Week 7 & \begin{minipage}{.85\textwidth}
%\begin{itemize} \itemsep-0.4em
%	\vspace{1mm}
%	\item Something interesting
%	\item Reading assignment: Something interesting
%	\vspace{1mm}
%\end{itemize}
%\end{minipage} \\
%\hline
%Week 8 & \begin{minipage}{.85\textwidth}
%\begin{itemize} \itemsep-0.4em
%	\vspace{1mm}
%	\item Something interesting
%	\item Midterm Exam
%	\vspace{1mm}
%\end{itemize}
%\end{minipage} \\
%\hline
%Week 9 & \begin{minipage}{.85\textwidth}
%\begin{itemize} \itemsep-0.4em
%	\vspace{1mm}
%	\item Something interesting
%	\item Reading assignment: Something interesting
%	\vspace{1mm}
%\end{itemize}
%\end{minipage} \\
%\hline
%Week 10 & \begin{minipage}{.85\textwidth}
%\begin{itemize} \itemsep-0.4em
%	\vspace{1mm}
%	\item Something interesting
%	\item Reading assignment: Something interesting
%	\vspace{1mm}
%\end{itemize}
%\end{minipage} \\
%\hline
%Week 11 & \begin{minipage}{.85\textwidth}
%\begin{itemize} \itemsep-0.4em
%	\vspace{1mm}
%	\item Something interesting
%	\item Reading assignment: Something interesting
%	\vspace{1mm}
%\end{itemize}
%\end{minipage} \\
%\hline
%Week 12 & \begin{minipage}{.85\textwidth}
%\begin{itemize} \itemsep-0.4em
%	\vspace{1mm}
%	\item Something interesting
%	\item Reading assignment: Something interesting
%	\vspace{1mm}
%\end{itemize}
%\end{minipage} \\
%\hline
%Week 13 & \begin{minipage}{.85\textwidth}
%\begin{itemize} \itemsep-0.4em
%	\vspace{1mm}
%	\item Something interesting
%	\item Reading assignment: Something interesting
%	\vspace{1mm}
%\end{itemize}
%\end{minipage} \\
%\hline
%Week 14 & \begin{minipage}{.85\textwidth}
%\begin{itemize} \itemsep-0.4em
%	\vspace{1mm}
%	\item Something interesting
%	\item Reading assignment: Review for Final Exam
%	\vspace{1mm}
%\end{itemize}
%\end{minipage} \\
%\hline
%\end{tabular} 
%\end{table}

\end{document}



